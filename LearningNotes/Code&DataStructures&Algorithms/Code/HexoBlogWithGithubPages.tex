\chapter{Hexo+Github Pages搭建个人博客}

\section{环境配置}
\subsection{安装NodeJS和Yarn}
注意nodejs不要安装默认附带的npm包管理工具。其他都是一路默认即可,包括yarn。

若想使用npm来管理包,请查看yarn和npm相关对应的命令。
\begin{lstlisting}[language={bash},title={yarn和npm对应命令}]
    yarn add # npm install
    yarn add [package] # npm install [package] --save
    yarn global add [package] # npm install [package] --global
\end{lstlisting}

\noindent {\footnotesize \emph{注 :Node版本选择v12,避免Stylus for Node v14 'Accessing non-existent property' errors。}}

\subsection{Yarn源设定}
\begin{lstlisting}[language={bash},title={\textsf{Yarn源设定命令}}]
    yarn config get registry  // 查看yarn当前镜像源
    yarn config set registry https://registry.npm.taobao.org/  // 设置yarn镜像源为淘宝镜像
\end{lstlisting}

\subsection{安装hexo}
在想要安装hexo的位置建立文件夹「hexo」,右键该文件夹,选择Git Bash。然后执行下面命令(推荐全局安装)。

\noindent {\footnotesize \emph{注:yarn add(hexo init之后)该条命令为可选,因为hexo init之后一般都会自动安装相关依赖。若没有自动安装时,则需要执行该条命令以手动安装依赖。}}

\begin{lstlisting}[language={bash},title={全局安装hexo}]
    yarn global add hexo-cli # 全局安装hexo-cli脚手架
    hexo init # 初始化hexo,克隆hexo-starter和默认landscape主题仓库
    hexo s # hexo server,启用本地服务器,见:http://localhost:4000/
\end{lstlisting}

\begin{lstlisting}[language={bash},title={本地安装hexo}]
    yarn add hexo-cli
    node_modules/.bin/hexo.cmd init blog # init需要空文件夹,所以另外用文件夹「blog」来初始化hexo
    cd blog
    yarn add
    node_modules/.bin/hexo.cmd s
\end{lstlisting}

\subsection{清除yarn已安装的包}
\begin{lstlisting}[language={bash},title={命令查看yarn相关路径}]
    yarn global dir # 全局安装路径
    yarn cache dir # 缓存路径,yarn缓存不会自动删除,省去重复下载
    yarn cache clean # 清理缓存路径
\end{lstlisting}

\section{Config配置}

\begin{intro}
    注意:以下代码中,cd hexo含义均代表:命令执行路径为站点的根目录,即为hexo init命令初始化的路径。以hexo为开头的路径均为:站点根目录/theme/next。「新增相关代码」若无提到位置,则任意位置都可以。
\end{intro}

\subsection{站点Config}
\subsubsection{本地实时预览 - browsersync}
\begin{lstlisting}[language={bash},title={安装插件hexo-browsersync}]
    cd hexo
    yarn add hexo-browsersync
\end{lstlisting}

\subsubsection{网站设置}
\noindent {\footnotesize \emph{注:language根据主题设置,注意新版Next主题已经没有zh-Hans语言配置文件,设置中文简体请用zh-CN参数。}}
\begin{lstlisting}[language={PHP},title={搜索站点Config.yml相关代码并修改}]
    # Site
    title: Mr. Kin's Blog
    subtitle: 健先生的博客
    description: Since you have chosen to chase, don't give up.
    author: Mr. Kin
    language: zh-CN
    timezone: Asia/Shanghai
\end{lstlisting}

\subsubsection{网址设置}
\begin{lstlisting}[language={PHP},title={搜索站点Config.yml相关代码并修改}]
    # URL
    url: https://mister-kin.github.io/
    permalink: :title/ # 仅使用title以优化seo
\end{lstlisting}

\subsubsection{使用第三方主题:Next}
\noindent {\footnotesize \emph{注:新版Next主题主仓库已从 iissnan 名下迁移至 theme-next 组织。}}

\begin{lstlisting}[language={bash},title={下载安装Next主题}]
    cd hexo
    git clone https://github.com/theme-next/hexo-theme-next themes/next
\end{lstlisting}

\begin{lstlisting}[language={PHP},title={搜索站点Config.yml相关代码并修改}]
    # Extensions
    theme: next
\end{lstlisting}

\subsubsection{Github Pages部署配置}
\begin{lstlisting}[language={PHP},title={搜索站点Config.yml相关代码并修改}]
    # Deployment
    deploy:
        type: git
        repo: git@github.com:mister-kin/mister-kin.github.io.git
        branch: master
\end{lstlisting}

\subsection{主题Config}
\subsubsection{选择方案「Scheme」}
\begin{lstlisting}[language={PHP},title={搜索主题Config.yml相关代码并修改}]
    # Schemes
    #scheme: Muse # 注释掉不使用的默认方案
    scheme: Gemini
\end{lstlisting}

\subsubsection{设置「语言」「作者昵称」「站点描述」}
见

\subsubsection{设置「菜单」}
\noindent {\footnotesize \emph{注:现阶段新版Next支持Font Awesome 5.13.0,而5.1.4旧版Next版本只支持Font Awesome 4.7.0}}
\begin{lstlisting}[language={PHP},title={搜索主题Config.yml相关代码并修改}]
    # Menu Settings
    menu:
        home: / || fa fa-home
        about: /about/ || fa fa-user
        tags: /tags/ || fa fa-tags
        logs: /logs || fa fa-file-alt
        commonweal: /404/ || fa fa-spinner
\end{lstlisting}
\begin{lstlisting}[language={PHP},title={搜索next/languages/zh-CN.yml相关代码并修改}]
    menu:
        logs: 日志
        commonweal: '404'
\end{lstlisting}

\subsubsection{设置「头像」}
\noindent {\bfseries \sffamily 加载头像}
在next/source/uploads/路径下新建文件夹「uploads」,并将头像放置于内。
\begin{lstlisting}[language={PHP},title={搜索主题Config相关代码并修改}]
    avatar: /uploads/avatar.webp
\end{lstlisting}
\noindent {\bfseries \sffamily 圆形化头像并旋转}
\begin{lstlisting}[language={PHP},title={next/source/css/main.styl底部新增相关代码}]
    // Customized Contents
    // 导入_custom/custom.styl
    @import "_custom/custom";
\end{lstlisting}

在next/source/css路径下新建文件夹「\_custom」,并新建custom.styl放置于内。
\begin{lstlisting}[language={PHP},title={next/source/css/\_custom/custom.styl新增相关代码}]
    // Customized Contents
    // 圆形化头像并旋转
    @import "../../uploads/css/avatar-round-rotate.styl"
\end{lstlisting}

在next/source/uploads/路径下新建文件夹「css」,并新建avatar-round-rotate.styl放置于内。
\begin{lstlisting}[language={PHP},title={next/source/uploads/avatar-round-rotate.styl新增相关代码}]
    // 配置文件为\themes\next\source\css\_custom\custom.styl。
    // 圆形化头像并旋转

    // 圆形化头像
    if (hexo-config('avatar_custom.round')) {
        .site-author-image {
        border-radius: 50%; // 大于50%效果都一样
        box-shadow: inset 0 -1px 0 #000000;
        }
    }

    // 鼠标经过头像旋转360度
    if (hexo-config('avatar_custom.rotate')) {
        .site-author-image {
        transition: transform 2s ease-out;
        }
        .site-author-image:hover {
        transform: rotateZ(360deg);
        }
    }
\end{lstlisting}

\subsubsection{设置RSS}
\begin{lstlisting}[language={bash},title={安装插件hexo-generator-feed}]
    cd hexo
    yarn add hexo-generator-feed
\end{lstlisting}

\subsubsection{设置代码高亮}
\begin{lstlisting}[language={PHP},title={搜索主题Config相关代码并修改}]
    codeblock:
        highlight_theme: night
        copy_button:
        enable: true
\end{lstlisting}

\subsubsection{设置社交链接}
\begin{lstlisting}[language={PHP},title={搜索主题Config相关代码并修改}]
    # Social links
    social:
\end{lstlisting}


\subsubsection{添加本地搜索 - Local Search}
\begin{lstlisting}[language={bash},title={安装插件hexo-generator-searchdb}]
    cd hexo
    yarn add hexo-generator-searchdb
\end{lstlisting}
\begin{lstlisting}[language={PHP},title={站点Config新增相关代码}]
    search:
        path: search.xml
        field: post
        format: html
        limit: 10000
\end{lstlisting}
\begin{lstlisting}[language={PHP},title={搜索主题Config相关代码并修改}]
    # Local search
    local_search:
        enable: true
\end{lstlisting}

去除推荐阅读(友链),开启fancybox,rss已经是放置于每篇文章背后
