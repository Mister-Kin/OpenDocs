\chapter{笔记本基础}
\begin{intro}
    笔记本基础,板号,架构,名词解释。内容来源见
\end{intro}
\section{整体介绍}
\begin{tabular}{|*{5}{c|}}
    \hline
    外形 & A壳 & B壳 & C壳 & D壳 \\
    \hline
    位置 & 顶面的壳 & 屏幕面的外框 & 键盘面的壳 & 底面的壳 \\
    \hline
\end{tabular}

\section{部件介绍}
\subsection{PCB(印刷电路板)}

笔记本PCB集成度高,一般6层以上,比如6(较少),8,10\dots,不像1或2层板,无法跑线。板层数越多,EMI性能越好,成本也越高。

PCBA是指PCB上装配按照设定规则指定元件的成品功能板。(元件:电阻,电容,芯片,接口\dots)

注:PCB, Printed Circuit Board; PCBA, Printed Circuit Board Assembly; EMI, Electromagnetic Interference

\subsection{Chipset(芯片组)}
一般指南北桥。(目前北桥已逐渐淘汰)

常见厂商有:AMD,INTEL,NVDIA,VIA,SIS。

南桥:North Bridge Chipset,INTEL的为输出/输入控制器中心(Input/Output Controller Hub,ICH),NVIDIA的为MCP,ATI的称为IXP/SB,AMD的为FCH。
北桥:North Bridge Chipset,INTEL的为GMCH,Graphics \& Memory controller hub,带G有集显,无G的无集显。

\subsection{CPU(中央处理器)}
常见厂商:AMD,INTEL。龙芯,VIA,IBM,Transmeta。

不同芯片组对应不同的Intel CPU座(阵脚不同)

注:CPU, Center Processor Unit

\subsection{Battery(电池)}
组成:外壳+控制板+电芯。
类型:圆柱型、方型、聚合物
电芯:常见18650型锂离子电芯。单个电压3.7V,充电电压4.2V,电容量为2400mAH。三串两并:电压为3*3.7V,容量为2*2400mAH。

\subsection{Adapter(适配器)}
输入100~240V的AC(50/60Hz),输出16~20V居多。(华硕EPC有9.5V和12V的输出)

\section{笔记本和台式机的区别}
\begin{itemize}
    \item 笔记本自带显示系统(LCD/LED,专用屏线接口,自带高压板,灯管)
    \item 笔记本电源统一只由一个电压转出,常见16~20。台式主板由ATX电源提供12V、5V、3.3V等电压。\\这个是最大区别,笔记本的工作电压是由板子转换完成(16-20V主供电输入,经PWM电路降压处理,提供待机电压等工作电压),台式主板电源完成(多种方式,不只PWM)。
    \item 笔记本有充放电的电路,可用电池做替代电源。
    \item 为保证笔记本移动性和续航,CPU低功率、节能设计。
    \item 笔记本的保护电路多,过热保护,过电压保护。
    \item 笔记本内置鼠标设备,如指点杆,触摸板。
    \item 笔记本元件集成度高,MOS管多为8脚贴片。专用IC也多。(芯片功能作用原理类似,供电不是太复杂。)
    \item 笔记本6-8层板,夹层中也有信号线。台式主板4层,一般只在正反面有信号线。
    \item 笔记本引入EC(多功能芯片)概念,类似台式主板的IO,但功能更多,因为处理键盘的各种信号(亮度调节,声音调节等快捷键)。部分EC里会带有程序,其脚位功能由程序决定。
    \item 笔记本的时序概念很重要,电压和功能的实现,都由时序控制。环环相扣,前面条件未完成,后面动作就不会执行。
    \item 笔记本维修对电路图依赖很强,需要电路图分析陌生元件,且需要点位图对照。无这些的话,只能维修一些通病。(信号复杂,板子小,整合度高)
\end{itemize}

\section{笔记本板号}
板号:板子型号。即工程代号,Project Code

笔记本大规模的代工厂:广达(quanta),仁宝(compal),纬创(wistron),英业达(inventec),和硕联合(pegatron)。
二线代工厂:神达(mitac),蓝天(clevo),大众(fic),微星(msi),精英(ecs)

OEM代工:品牌商设计,代工厂生产。如苹果,联想(thinkpad)。成本高
ODM代工:设计和生产都是代工厂。

广达:DAO+板号+mb,一般为3个字,如ch3,zq5
仁宝:la-xxxxp
纬创:板号+mb(有白框)
华硕(asus): 板号+main board(没有位置号,PCB丝印层无标记,若无点位图无法分析)
英业达:板号(很长)+mb,一般给hp做得多
微星:ms-板号
富士康:代工索尼

\section{主板板子元件}
CPU座,北桥,南桥,内存插槽,独立显卡,显存,SPI BIOS,pice,电池接口,适配器插头,时钟芯片,ec,LCD接口,硬盘接口,键盘接口,光驱接口,读卡器槽。

\section{笔记本主板架构}
修接口。供电维修看架构没用。

intel双桥架构:
\begin{enumerate}
    \item CPU管理北桥。
    \item 北桥管内存,独显,显示接口,与cpu的连接的总线---FSB前端总线。北桥与南桥的总线---DMI+CLINK
    \item 南桥管理周边设备,网卡,迷你卡,USB,摄像头,EC,光驱,硬盘等,南桥和EC连接的总线--LPC总线,7根重要信号:LAD0,LAD1,LAD2,LAD3,LFRAME\#,LCLK,LRESET\#。(诊断卡接9根,外加VCC、GND)
    \item EC管理键盘,触控板,鼠标,部分挂BIOS(SPI ROM)
\end{enumerate}

intel单桥架构(无北桥):
\begin{enumerate}
    \item CPU管理显卡,内存。CPU不管显示接口,通过PCH桥到显示接口(cpu里有集成显卡,通过FDI总线输出)
    \item pch(管理原来南桥的功能)相比原南桥,增加了显示接口(VGA,LVDS,HDMI等)管理,可能也直接管理SPI rom(BIOS)
    \item EC管理键盘,温控芯片,触摸板,挂BIOS
\end{enumerate}

AMD(ATI)双桥架构:
\begin{enumerate}
    \item CPU管内存
    \item 北桥管理所有PCIE,显示接口。网卡(1000M,走PCIE)
    \item 南桥管理USB,网卡(100M,走PCI。),硬盘光驱等
\end{enumerate}

AMD单桥架构:
\begin{enumerate}
    \item CPU管理内存,显卡,显示接口(这个是与intel的区别)
    \item 桥(fch)管理网卡,mini-pcie,硬盘光驱,USB,EC,声卡等
    \item EC管键盘,鼠标
\end{enumerate}

intel和AMD单桥架构无时钟芯片,集成在桥。nvidia的双桥和amd相同。nvidia单桥:CPU管内存,桥管其他。

\section{名词解释}
\begin{intro}
    复位和PG都是测电压,时钟是测频率(无示波器时,可测电压,33MHz大概1.6V,100MHz大概0.4V)。
\end{intro}
\subsection{供电和信号}
\subsubsection{供电}
供电一个可以输出电流的电压,电流较大。工作过程中,这个电压不能置高或拉低。供电被拉低,就是短路。一般,也不许置高。(动力来源)。

常见有19,12,5(往上大电压给接口),3.3(给芯片),2.5,1.8,1.5,1.25,1.05,1.2,1.1,0.9,0.75V。CPU供电0.7-1.5V

名称一般为:VCC,VDD,VCC3,VDDQ,VTT,VBAT,5VALW,+3VO等(有V字)。苹果的供电特殊,例如PP0V75\_s3\_mem

符号为一个圆圈,T型,三角形。表示固定的电压,不允许置高和拉低。信号电压(例如19V)与1.5V(供电)碰在一起会变为1.5V。

\subsubsection{接地}
接地是给供电构成回路。有接地,才会有电流流过设备。

名称一般为:VSS、GND。

符号:三角形(数字地);倒三角形多横线(模拟地)。避免数字和模拟连在一起相互干扰。例子:数字地和模拟地通过一个0欧姆电阻PR170(值0\_6),实际测量是通路,但信号不一样。如果这个0欧姆电阻坏了,可能导致烧元件。压差相对值不一样。

\subsubsection{信号}
理论上,电压信号值考虑电压变化,电流很小。在工作过程中,可根据需要置高或者拉低。电路图中的信号的箭头不完全代表信号的流向,因为画图人员的随意性。
信号方向考经验判断:例如PWRBTN\#给南桥的;slp\_s3\#南桥出来的;therm\_stp\#温控信号看情况:过温时,温控芯片吧温控信号拉低。

\subsection{高低电平}
数字逻辑电路中,高电平表示1,低电平表示0。一般规定:低电平为0-0.25V,高电平为3.5-5V。

主板中,1V以上为高电平,0.7以下为低电平。

结论:根据电路判断高低电平,非限定特定值。有些电路0.5就是高,有的电路1.1还是低。但0肯定是地,3.3肯定是高。

\subsection{脉冲和跳变}
上升沿,下降沿。

类型:高电平跳变为低电平;低电平跳变为高电平;高跳变为低再跳变为高。

\subsection{时钟信号}
时钟信号CLK(Clock的简写)。为数字电路工作提供一个基准,使各个相连设备统一步调工作。单位Hz。南桥晶振323.768KHz。

主板上都有一个主时钟产生电路,给所有设备提供时钟,送出到cpu的频率为100MHz以上,送给PCI的是33MHz,送给PCIE的是100MHz,送给USB控制器(集成在南桥内部)的为48MHz。

相连的两个设备需要相同的时钟频率和电压才能通信,如内存和北桥。

时钟信号需要在主板正常通电后且时钟芯片工作正常才能测量到,用示波器和万用表(测电压?)测。100M的示波器一般测不了CPU的频率。

\subsection{复位信号}
复位信号RST(RESET的简写)。复位都是从高电平向低电平跳变再回到高电平,如PCI的复位是从3.3V向0V跳变再回到3.3V就是一个正常的复位跳变。

名称一般为:xxxRST\#,如PCIRST\#、CPURST\#、IDERST\#等。复位只能是瞬间低电平,主板正常工作时复位都是高电平。但不是恒高电平,不能直接接到供电上。如台式机reset键,复位开关弹不起来就一直为低电平,就不行。

平常说没复位,通常指没复位电压,即复位信号测量点电压为0V。维修一般都是把复位电压修出来。

所有设备的复位信号,如EC,北桥等,都是由南桥发出。开机的瞬间,便会对设备清零,使其重新工作。

\subsection{PG信号}
电源好信号PG(powergood的缩写),用来描述供电正常的信号。一般高电平有效。如cpu供电芯片,只有在正常发出cpu电压后,才会发出PG信号。

名称一般为:PG、PWRGD、PWROK、POK、PWRG、VTTPWRGD、CPUPWRGD等。

\subsection{开启信号}
开启信号。有的芯片叫EN(enable),使能,高电平时表示开启信号。有的芯片叫SHDN\#(shutdown),\#表示低电平有效。综合名称和\#来看,意思是低电平时关闭芯片,高电平开启。所以一般shutdown信号一般要维持高电平。

注:信号带\#时(低电平有效),一定要结合信号的英文全程去理解。有的带\#,为低电平时主板可以正常工作。例如:VT\_PWRGD\_CK410\#信号是cpu供电正常后发出低电平开启时钟信号。1999\_SHDN\#信号是低电平关闭MAX1999的控制信号,即为高电平时,主板才能正常工作。
