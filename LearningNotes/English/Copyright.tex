% !TEX root = English.tex %指定主文件。
\ifx\collections\undefined
% !TEX program = xelatex %指定编译方式xelatex。
% !BIB program = biber %指定bib数据后台处理程序biber。
\documentclass[11pt,a4paper,UTF8,titlepage]{ctexrep} %指定ctexart文档类,设置基本字号为11pt,a4大小,使用UTF-8编码保存,指定\maketitle生成单独的标题页。

\usepackage{syntonly}
%\syntaxonly %用来快速编译以排查错误,不生成DVI或PDF。
\usepackage[style=gb7714-2015]{biblatex} %调用biblatex宏包,设置参考文献样式(符合中文文献著录标准GB/T 7714-2015的样式),使用默认的后端程序biber(其支持更好,包括UTF-8等)放弃使用[backend=bibtex](只支持ascii编码)。
\addbibresource{resources/reference.bib} %加载参考文献数据库。
\usepackage{makeidx}
\makeindex %开启索引收集
\usepackage[margin=1in]{geometry} %调用geometry宏包,设置周围页边距为1英寸(为电子档设计,非打印)。
\usepackage{xcolor} %调用xcolor宏包,以支持扩展生成颜色。
\usepackage{fontspec} %调用fontspec宏包以更改西文字体族。
\setmainfont{Source Serif Pro}
\setsansfont{Source Sans Pro}
\setmonofont{Source Code Pro}
\usepackage{xeCJK} %调用xeCJK宏包以更改中文字体族。
\xeCJKsetup{AutoFakeSlant=true}  %设置xeCJK选项-伪斜体(需置于字体设置之前)。
\setCJKmainfont{思源宋体}
\setCJKsansfont{思源黑体}
\setCJKmonofont{思源等宽}
\usepackage{graphicx} %调用graphicx宏包,以支持插图。
\graphicspath{{resources/images/},{resources/images/FollowMe/}} %加载图片路径。
\usepackage{caption} %调用caption,支持不带编号的标题。
\usepackage{wrapfig} %调用wrapfig,支持图文排版。
\usepackage{subfigure} %调用subfigure宏包进行图片图片排版。
\usepackage{tikz,tikz-qtree} %调用tikz及其扩展宏包,以支持画图。
\usepackage[subfigure]{tocloft} %tocloft与subfigure宏包冲突,不能简单调用,tocloft需设置参数。
\usepackage{tocbibind} %调用宏包以添加目录本身和参考文献进目录中。
\usepackage{multicol} %调用multicol宏包以支持多栏排版。
\usepackage[toc]{multitoc} %调用multitoc宏包,设置toc目录页,默认双栏排版。
\usepackage{enumitem} %调用emuitem,以设置列表环境。
\usepackage{multirow} %调用multirow,以支持纵向合并列表。
%重订制目录命令。
\renewcommand{\tableofcontents}%
  {\chapter{\contentsname}%
  \@mkboth{\MakeUppercase\contentsname}{\MakeUppercase\contentsname}%
  \@makeschapterhead{\sourcecodename}%
  \@starttoc{toc}%
}
\usepackage{fancyhdr} %调用宏包,以设置页眉,统一格式:标题在左,页码在右。
\pagestyle{fancy}
\fancyhf{}
\fancyhead[LO]{\sffamily \rightmark}
\fancyhead[LE]{\sffamily \leftmark}
\fancyhead[ROE]{\bfseries \thepage}
\fancyfoot[COE]{\ttfamily \href{https://mister-kin.github.io}{个人博客:https://mister-kin.github.io}}
%定义intro环境。
\newenvironment{intro}{\narrower\sffamily}{\par\vspace*{2ex plus 2.5ex minus 1.5ex}}
\usepackage{listings} % 指定listings,订制代码排版环境
\lstset{
    basicstyle      = \ttfamily,                          % 基本代码指定等宽字体
    keywordstyle    = \bfseries,                          % 关键字指定加粗
    commentstyle    = \ttfamily\slshape\color{gray},      % 注释指定灰色等宽斜体
    stringstyle     = \ttfamily,                          % 字符串指定等宽字体
    %numbers        = left,                               % 行号的位置在左边,启用后不方便复制代码
    %numberstyle    = \ttfamily,                          % 行号等宽字体
    %xleftmargin    = \parindent,                         % 代码左边框起始位置(启用行号时建议启用这个)
    %frame          = trBL,                               % 代码框类型,t下,r右,b下,l左,大写时为两条线。
    %frameround     = fttt,                               % 控制代码框是否为圆角
    frameshape      = {{ryrynyyyy}{yny}{yny}{ryrynyyyy}}, % 控制边框样式,上下边是每三个字母段控制一条边框。
    backgroundcolor = \color{gray!5},                     % 代码框背景颜色:5%的灰色
    breaklines      = true,                               % 代码过长时则换行
    gobble          = 8,                                  % 去掉代码前的缩进
}
\usepackage{hyperref} %调用hyperref宏包
\hypersetup{
    colorlinks=true, %设置超链接文件带颜色
    bookmarks=true, %生成书签
    bookmarksopen=true, %书签展开
    bookmarksnumbered=true, %书签带章节编号
    CJKbookmarks=true, %cjk必设参数
    unicode, %utf-8编码必设参数
    pdftitle=标题, %设置PDF文件属性标题
    pdfauthor=Mr. Kin, %设置PDF文件属性作者
    pdfstartview=FitH %默认适合宽度显示
}

    \title{\hypertarget{title}{\textbf{标题}}}
    \addcontentsline{toc}{chapter}{标题页}
    \author{Written by Mr. Kin}
    \date{创建于2020.1.28,修改于\number\year.\number\month.\number\day}

\begin{document}
    \phantomsection %确保目录中的超链接指向正确的页码
    \pdfbookmark[1]{标题页}{title} %添加标题页书签
    \pagenumbering{Roman} %大写罗马样式页码
    \maketitle %生成标题页
    \pagenumbering{roman} %小写罗马样式页码
    {\centering \tableofcontents} %生成目录页
    \clearpage %新建页,分离上下两个样式页码的效果
    \pagenumbering{arabic} %阿拉伯样式页码
    \fi

    \phantomsection
    \section*{\bfseries \sffamily 版权声明}
    \addcontentsline{toc}{chapter}{版权声明}
    \markright{版权声明}
    \noindent 作者:Mr. Kin \\
    博文链接:\href{}{跳转博文页}\\
    %相关视频创作链接:\href{}{跳转视频页}\\ %无视频则注释掉此行
    PDF及Tex源码链接:\href{https://github.com/Mister-Kin/OpenDocs/tree/master/LearningNotes/English}{跳转PDF及Tex源码页}\\
    许可协议:本作品的所有内容,除个人设计创作的图像(如logo等)和相关的视频创作及其他特别声明外,均采用\href{https://creativecommons.org/licenses/by-nc-sa/4.0/deed.zh}{知识共享\ 署名-非商业性使用-相同方式共享 4.0 国际许可协议}进行发布。版权 © Mr. Kin,保留所有权利。\includegraphics[scale=.4]{CC-BY-NC-SA}\\*[1.3ex]
    \begin{tabular}{|*{3}{p{0.306\textwidth}|}}
        \hline
        \textsf{\bfseries 允许} & \textsf{\bfseries 限制} & \textsf{\bfseries 条件} \\
        \hline
        \vspace{-8pt}{\color{green}√} 修改 & \vspace{-8pt}{\color{red}×} 商标使用 & \vspace{-8pt}{\color{blue}$\odot$} 保留原署名 \\[-12pt]
        {\color{green}√} 分发 & {\color{red}×} 专利使用 & {\color{blue}$\odot$} 状态变更说明 \\[-12pt]
        {\color{green}√} 个人使用 & {\color{red}×} 商业使用 & {\color{blue}$\odot$} 相同的许可和版权声明 \\
        \hline
    \end{tabular}
    \\*[1.3ex]
    \emph{注:若想对本作品进行转载、引用亦或是进行二次创作时,请详细阅读上述相关协议内容(若不理解,请点击链接跳转阅读)。为保障本人权利,对于违反者,本人将依法予以处理!望周知!——Mr. Kin}
    \section*{\bfseries \sffamily 勘误声明}
    虽本人写作时已尽力保证其内容的正确性,但因个人知识面和经验的局限性以及计算机技术等相关技术日新月异,本作品内容或存在一些错误之处。还望诸君发现错误后能够\hyperlink{contact}{联系我}以更正错误,不胜感激!——Mr. Kin

    \section*{\bfseries \sffamily 侵权声明}
    若本作品采用的第三方内容侵犯了你的版权,请与我\hyperlink{contact}{联系}进行处理,谢谢!——Mr. Kin

    \section*{\bfseries \sffamily 第三方开源许可声明}
    本作品使用的第三方开源产品有:
    \begin{multicols}{2}
    \begin{itemize}
        \item \href{https://github.com/adobe-fonts}{Adobe Fonts}: \href{http://scripts.sil.org/cms/scripts/page.php?site_id=nrsi&id=OFL}{OFL v1.1}
        \item \href{https://tug.org/texlive/}{Tex Live}: \href{https://tug.org/texlive/copying.html}{TeX Live Licensing}
        \item \href{https://code.visualstudio.com/}{Visual Studio Code}: \href{https://www.mit-license.org/}{MIT}
        \item \href{http://ffmpeg.org/}{FFmpeg}: \href{http://ffmpeg.org/legal.html}{LGPL v2.1}
        \item \href{https://krita.org/en/}{Krita}: \href{https://docs.krita.org/en/KritaFAQ.html?highlight=license#license-rights-and-the-krita-foundation}{Krita's GPL license}
        \item \href{https://inkscape.org/}{Inkscape}: \href{https://inkscape.org/about/license/}{GPL}
        \item \href{https://www.gimp.org}{GIMP}: \href{https://www.gimp.org/about/COPYING}{GPL}
        \item \href{https://www.blender.org}{Blender}: \href{https://www.blender.org/about/license/}{GPL}
        \item \href{https://www.audacityteam.org/}{Audacity}: \href{https://www.gnu.org/licenses/old-licenses/gpl-2.0.en.html}{GPL v2}
        \item \href{https://handbrake.fr}{Handbrake}: \href{https://github.com/HandBrake/HandBrake/blob/master/LICENSE}{GPL v2}
    \end{itemize}
    \end{multicols}

    \ifx\collections\undefined
    \printbibliography %生成参考文献排版。
    \addcontentsline{toc}{chapter}{参考文献} %添加参考文献进目录
    \clearpage %新建页,确保超链接跳转正确
    \phantomsection %确保目录中的超链接指向正确的页码
    \printindex %生成索引排版。
\end{document}
    \fi
