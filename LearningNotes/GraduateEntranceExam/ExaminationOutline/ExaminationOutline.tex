\chapter{考试大纲}

\begin{intro}
    政治、英语、数学这三个公共课是全国统考,专业课只有法硕、农学、教育学、心理学、历史学、西医综合、计算机是全国统考,其他的都是学校自主出题。
\end{intro}

\noindent {\bfseries \sffamily 成绩线}
待查询。

\section{考试及试题信息}
\subsection{数学一}
\paragraph{试卷情况} 分值150,考试时长180min。
\paragraph{内容占比} 56\%高等数学(微积分),22\%线性代数,22\%概率论与数理统计。
\paragraph{题型结构:}
\begin{enumerate}
    \item 单项选择8道,共32分
    \item 填空题6道,共24分
    \item 解答题(含证明题)9道,共94分
\end{enumerate}

\subsection{英语一}
\paragraph{试卷情况} 分值100,考试时长180min。
\paragraph{内容}词汇(5500+附表词汇);语法(基本熟练地运用基本的语法知识)。
\paragraph{题型结构:}
\begin{enumerate}
    \item 英语知识运用(10分)
    \begin{enumerate}
        \item 20道完形填空(四选一):1篇文章(240~280词),词汇、语法和结构。
    \end{enumerate}
    \item 阅读理解(40+10+10分)
    \begin{enumerate}
        \item 20道多项选择(四选一):4篇文章(共约1600词),理解主旨要义、具体信息、概念性含义、进行判断、推理和引申,根据上下文推测生词词义。
        \item 5道选择搭配:1篇文章(500~600词),理解连贯性、一致性等语段特征和文章结构。
        \item 5道英译汉:1篇文章(400词,5个划线部分约150词),理解复杂概念、结构。
    \end{enumerate}
    \item 写作(10+20分)
    \begin{enumerate}
        \item 一个应用文写作(规定情景:信函、备忘录、报告等):100词。
        \item 一个短文写作(描述性、叙述性、说明性、议论性文章):160~200词。
    \end{enumerate}
\end{enumerate}

\subsection{政治}
\paragraph{试卷情况} 分值100,考试时长180min。
\paragraph{内容占比} 24\%马原,30\%毛中特,14\%史纲,16\%思修,16\%时政和当代。
\paragraph{题型结构:}
\begin{enumerate}
    \item 单项选择16道:16分
    \item 多项选择17道:34分
    \item 材料分析题:50分
\end{enumerate}

\subsection{408}
\paragraph{试卷情况} 分值150,考试时长180min。
\paragraph{内容占比} 数据结构(45分),计算机组成原理(45分),操作系统(35分),计算机网络(25分)。
\paragraph{题型结构:}
\begin{enumerate}
    \item 单项选择40道:80分
    \item 综合应用题:70分
\end{enumerate}

\section{复习规划}
学习纲要:学习完一节知识后进行复盘记录。

\subsection{数学}
\paragraph{复习资料} 线代李永乐、高数汤家凤/武忠祥,概率王式安。

\begin{enumerate}
    \item
\end{enumerate}

\subsection{政治}
\paragraph{复习资料} 《精讲精练》《1000题》《八套卷》《四套卷》《风中劲草》《模拟卷》

\subsection{408}
