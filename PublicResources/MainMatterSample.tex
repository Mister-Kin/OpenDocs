\part{字体测试}
\chapter{字体测试}
\section{中文字体测试}
\sampletext

\begin{table}[h]
    \centering
    \caption{字体样式测试}
    \begin{tabular}{|*{6}{c|}}
        \hline
        \diagbox{字体名称}{字体样式} & 普通           & 粗体                    & 倾斜体                 & 意大利斜体             & 直立体                 \\
        \hline
        思宋(rmfamily)             & 测试           & \bfseries 测试          & \slshape 测试          & \itshape 测试          & \upshape 测试          \\
        \hline
        思黑(sffamily)             & \sffamily 测试 & \sffamily\bfseries 测试 & \sffamily\slshape 测试 & \sffamily\itshape 测试 & \sffamily\upshape 测试 \\
        \hline
    \end{tabular}
\end{table}

{思宋:\sampletext}

{\sffamily 思黑:\sampletext}

\section{English Font Test}
\sampletexten

\begin{table}[h]
    \centering
    \caption{English Font Style Test}
    \begin{tabular}{|*{6}{c|}}
        \hline
        \diagbox{Font Name}{Font Style} & Normal         & Bold Face               & Slanted                & Italic                 & Upright                \\
        \hline
        Source Serif(rmfamily)          & test           & \bfseries test          & \slshape test          & \itshape test          & \upshape test          \\
        \hline
        Source Sans(sffamily)           & \sffamily test & \sffamily\bfseries test & \sffamily\slshape test & \sffamily\itshape test & \sffamily\upshape test \\
        \hline
        Source Code Pro(ttfamily)       & \ttfamily test & \ttfamily\bfseries test & \ttfamily\slshape test & \ttfamily\itshape test & \ttfamily\upshape test \\
        \hline
    \end{tabular}
\end{table}

{Source Serif: \sampletexten}

{\sffamily Source Sans: \sampletexten}

{\ttfamily Source Code Pro: \sampletexten}

\part{各类测试}
\chapter{章标题测试}

\begin{intro}
    intro环境测试。intro环境,用以章节开头的文字介绍。排版上比普通正文多缩进两个文字。\cite{citebook}
\end{intro}

正文环境测试\footnote{脚注测试。\sampletext}:\repeated{3}{\sampletext} \cite{citeonline,citemanual}

\section{节标题测试}
正文环境测试\footnote{脚注测试。\sampletext}:\repeated{2}{\sampletext}

\repeated{2}{\sampletext}

\subsection{子节标题测试}
正文环境测试:\repeated{4}{\sampletext}

\subsubsection{子子节标题测试}
正文环境测试\footnote{脚注测试。\sampletext}:\repeated{4}{\sampletext}

\paragraph{普通文字段} \repeated{3}{\sampletext}

\subparagraph{普通文字段} \repeated{3}{\sampletext}

\chapter{其余宏包和命令测试}
\begin{intro}
    本文类是基于ctexbook文类开发的,本章会简单地列举部分命令和宏包命令的使用效果。其余详细完整的命令使用请参阅本文类所加载的宏包的手册文档。
\end{intro}

\section{本文类所加载的宏包}
\begin{multicols}{4}
    \begin{itemize}
        \item fontspec
        \item xeCJK
        \item amsmath
        \item unicode-math
        \item geometry
        \item tocloft
        \item tocbibind
        \item multitoc
        \item biblatex
        \item graphicx
        \item fancyhdr
        \item hyperref
        \item listings
        \item enumitem
        \item caption
        \item wrapfig
        \item subfigure
        \item tikz
        \item tikz-qtree
        \item multirow
        \item booktabs
        \item array
        \item colortbl
        \item makecell
        \item diagbox
        \item longtable
        \item xcolor
        \item ulem
        \item multicol
        \item fontawesome
    \end{itemize}
\end{multicols}

\section{图片测试}
\sampletext

\begin{figure}[h]
    \centering
    \includegraphics[scale=0.95]{SampleImage}
    \caption{示例图片}
\end{figure}

\sampletext

\section{列表}
\subsection{默认有序列表}
\begin{enumerate}
    \item test
    \item test测试
    \item 测试
    \item 测试test
    \item \sampletexten
    \item \sampletext
    \item \sampletexten \sampletexten
\end{enumerate}

\subsection{默认无序列表}
\begin{itemize}
    \item test
    \item test测试
    \item 测试
    \item 测试test
    \item \sampletexten
    \item \sampletext
    \item \sampletexten \sampletexten
\end{itemize}

\subsection{自定义标签有序列表(enumitem宏包)}
\begin{enumerate}[label={Step \arabic*.}]
    \item test
    \item test测试
    \item 测试
    \item 测试test
    \item \sampletexten
    \item \sampletext
    \item \sampletexten \sampletexten
\end{enumerate}

\section{下划线测试(ulem宏包)}
\uline{下划线}
\uuline{双下划线}
\uwave{波浪线}
\sout{删除线}
\xout{斜线}
\dashuline{下划线-虚线}
\dotuline{下划线-点}

\section{代码环境测试(listings宏包)}
\sampletext

\begin{lstlisting}[language={C},title={\textsf{C语言代码段测试}}]
        for(int i=0,i>0,i++);
            printf("Hello World!"); // 注释测试 Comment Test
\end{lstlisting}

\sampletext

\section{自定义命令测试}
\subsection{IPA国际音标排版测试(English)}
anode \ipa{'ænoʊd}

cathode \ipa{'kæθoʊd}

\subsection{罗马数字测试}
\romannum{3}\ \Romannum{3} $K_{test}^{\text{\Romannum{3}}}$ $K_{test}^{\mathrm{\Romannum{3}}}$

\subsection{数学公式测试}
行内公式:$a^2 + b^2 = c^2$

行间公式
\begin{equation}
    a^2 + b^2 = c^2 \label{公式测试}
\end{equation}
式子\eqref{公式测试}……。

\subsection{颜色测试(xcolor宏包)}
\mytextcolor{blue}{蓝色} {\mycolor{red} 红色}

\note{为适配本宏包的 print 选项,颜色测试已用自定义命令。}

\subsection{note命令测试}
\note{note测试}

\rednote{测试note}

\enote{note测试}

\redenote{测试note}
