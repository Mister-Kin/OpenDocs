\documentclass{../../PublicResources/DocClass}

    \PathPublicResources{../../PublicResources}

    \DocumentTitle{文档标题测试}
    \DocumentSubtitle{文档副标题测试}
    \DocumentCreatedDate{2020/7/21}

    \LinkBlogPost{https://test.LinkBlogPost/}
    \LinkPDF{https://test.LinkPDF/}
    \LinkPDFAccessCode{docs}
    \LinkLaTeX{https://test.LinkLaTeX/}
    \LinkVideo{https://test.LinkVideo/}

    \AuthorName{Mr. Kin}
    \AuthorEmail{im.misterkin@gmail.com}
    \AuthorBlog{https://mister-kin.github.io/}

\begin{document}
    \maketitle
    \frontmatter
    \inputPubulicText
    \inputToc
    \mainmatter

    % 正文
    \part{中文字体测试}
    \chapter{章标题测试}
    \section{节标题测试}
    \subsection{子节标题测试}
    \subsubsection{子子标题测试}
    \paragraph{段标题测试}
    测试
    \subparagraph{子段标题测试}
    测试

    \section{字体样式测试}
    \begin{table}[h]
        \centering
        \begin{tabular}{|*{6}{c|}}
            \hline
            \diagbox{字体名称}{字体样式} & 普通 & 粗体 & 倾斜体 & 意大利斜体 & 直立体 \\
            \hline
            思宋(rmfamily) &  测试 & \bfseries 测试 & \slshape 测试 & \itshape 测试 & \upshape 测试 \\
            \hline
            思黑(sffamily) & \sffamily 测试 & \sffamily\bfseries 测试 & \sffamily\slshape 测试 & \sffamily\itshape 测试 & \sffamily\upshape 测试 \\
            \hline
        \end{tabular}
    \end{table}

    \section{字体排版测试}
    {思宋:滚滚长江东逝水,浪花淘尽英雄。是非成败转头空,青山依旧在,几度夕阳红。白发渔樵江渚上,惯看秋月春风。一壶浊酒喜相逢,古今多少事,都付笑谈中。}

    {\sffamily 思黑:滚滚长江东逝水,浪花淘尽英雄。是非成败转头空,青山依旧在,几度夕阳红。白发渔樵江渚上,惯看秋月春风。一壶浊酒喜相逢,古今多少事,都付笑谈中。}

    \part{English Font Test}
    \chapter{Chapter Title Test}
    \section{Section Title Test}
    \subsection{Subsection Title Test}
    \subsubsection{Subsubsection Title Test}
    \paragraph{Paragraph Title Test}
    Test
    \subparagraph{Subparagraph Title Test}
    Test

    \section{Font Style Test}
    \begin{table}[h]
        \centering
        \begin{tabular}{|*{6}{c|}}
            \hline
            \diagbox{Font Name}{Font Style} & Normal & Bold Face & Slanted & Italic & Upright \\
            \hline
            Source Serif(rmfamily) & test & \bfseries test & \slshape test & \itshape test & \upshape test \\
            \hline
            Source Sans(sffamily) & \sffamily test & \sffamily\bfseries test & \sffamily\slshape test & \sffamily\itshape test & \sffamily\upshape test \\
            \hline
            Source Code Pro(ttfamily) & \ttfamily test & \ttfamily\bfseries test & \ttfamily\slshape test & \ttfamily\itshape test & \ttfamily\upshape test \\
            \hline
        \end{tabular}
    \end{table}

    \section{Font Typography Test}
    {Source Serif: The quick brown fox jumps over the lazy dog}

    {\sffamily Source Sans: The quick brown fox jumps over the lazy dog}

    {\ttfamily Source Code Pro: The quick brown fox jumps over the lazy dog}

    \part{各类测试}
    \chapter{章标题测试}

    \begin{intro}
        intro环境测试。intro环境,用以章节开头的文字介绍。排版上比普通正文多缩进两个文字。
    \end{intro}

    正文环境测试\footnote{脚注测试}:滚滚长江东逝水,浪花淘尽英雄。是非成败转头空,青山依旧在,几度夕阳红。白发渔樵江渚上,惯看秋月春风。一壶浊酒喜相逢,古今多少事,都付笑谈中。

    \section{节标题测试}
    正文环境测试:滚滚长江东逝水,浪花淘尽英雄。是非成败转头空,青山依旧在,几度夕阳红。白发渔樵江渚上,惯看秋月春风。一壶浊酒喜相逢,古今多少事,都付笑谈中。

    \subsection{子节标题测试}
    正文环境测试:滚滚长江东逝水,浪花淘尽英雄。是非成败转头空,青山依旧在,几度夕阳红。白发渔樵江渚上,惯看秋月春风。一壶浊酒喜相逢,古今多少事,都付笑谈中。

    \subsubsection{子子节标题测试}
    正文环境测试:滚滚长江东逝水,浪花淘尽英雄。是非成败转头空,青山依旧在,几度夕阳红。白发渔樵江渚上,惯看秋月春风。一壶浊酒喜相逢,古今多少事,都付笑谈中。

    \paragraph{普通文字段} 滚滚长江东逝水,浪花淘尽英雄。是非成败转头空,青山依旧在,几度夕阳红。白发渔樵江渚上,惯看秋月春风。一壶浊酒喜相逢,古今多少事,都付笑谈中。

    \subparagraph{普通文字段} 滚滚长江东逝水,浪花淘尽英雄。是非成败转头空,青山依旧在,几度夕阳红。白发渔樵江渚上,惯看秋月春风。一壶浊酒喜相逢,古今多少事,都付笑谈中。

    \chapter{其余宏包和命令测试}
    \begin{intro}
        本文类是基于ctexbook文类开发的,本章会简单地列举部分命令和宏包命令的使用效果。其余详细完整的命令使用请参阅本文类所加载的宏包的手册文档。
    \end{intro}

    \section{本文类所加载的宏包}
    \begin{multicols}{4}
        \begin{itemize}
            \item fontspec
            \item xeCJK
            \item amsmath
            \item unicode-math
            \item geometry
            \item tocloft
            \item tocbibind
            \item multitoc
            \item biblatex
            \item graphicx
            \item fancyhdr
            \item hyperref
            \item listings
            \item enumitem
            \item caption
            \item wrapfig
            \item subfigure
            \item tikz
            \item tikz-qtree
            \item multirow
            \item booktabs
            \item array
            \item colortbl
            \item makecell
            \item diagbox
            \item longtable
            \item xcolor
            \item ulem
            \item multicol
            \item fontawesome
        \end{itemize}
    \end{multicols}

    \section{图片测试}
    \begin{figure}[h]
        \centering
        \includegraphics[scale=0.8]{SampleImage}
        \caption{示例图片}
    \end{figure}

    \section{列表}
    \subsection{默认有序列表}
    \begin{enumerate}
        \item test
        \item test测试
        \item 测试
        \item 测试test
        \item 滚滚长江东逝水,浪花淘尽英雄。是非成败转头空,青山依旧在,几度夕阳红。白发渔樵江渚上,惯看秋月春风。一壶浊酒喜相逢,古今多少事,都付笑谈中。
    \end{enumerate}

    \subsection{默认无序列表}
    \begin{itemize}
        \item test
        \item test测试
        \item 测试
        \item 测试test
        \item 滚滚长江东逝水,浪花淘尽英雄。是非成败转头空,青山依旧在,几度夕阳红。白发渔樵江渚上,惯看秋月春风。一壶浊酒喜相逢,古今多少事,都付笑谈中。
    \end{itemize}

    \subsection{自定义标签有序列表(enumitem)}
    \begin{enumerate}[label={Step \arabic*.}]
        \item test
        \item test测试
        \item 测试
        \item 测试test
        \item 滚滚长江东逝水,浪花淘尽英雄。是非成败转头空,青山依旧在,几度夕阳红。白发渔樵江渚上,惯看秋月春风。一壶浊酒喜相逢,古今多少事,都付笑谈中。
    \end{enumerate}

    \section{下划线测试(ulem)}
    \uline{下划线}
    \uuline{双下划线}
    \uwave{波浪线}
    \sout{删除线}
    \xout{斜线}
    \dashuline{下划线-虚线}
    \dotuline{下划线-点}

    \section{代码环境测试(listings)}
    \begin{lstlisting}[language={C},title={\textsf{C语言代码段测试}}]
        for(int i=0,i>0,i++);
            printf("Hello World!"); // 注释测试 Comment Test
    \end{lstlisting}

    \section{自定义命令测试}
    \subsection{IPA国际音标排版测试(English)}
    anode \ipa{'ænoʊd}

    cathode \ipa{'kæθoʊd}

    \subsection{罗马数字测试}
    \romannum{3}\ \Romannum{3} $K_{test}^{\text{\Romannum{3}}}$ $K_{test}^{\mathrm{\Romannum{3}}}$ $\Delta Q$

    \subsection{颜色测试(xcolor)}
    \mytextcolor{blue}{蓝色} {\mycolor{red} 红色}

    \note{为适配本宏包的 print 选项,颜色测试已用自定义命令。}

    \subsection{note命令测试}
    \note{note测试}

    \rednote{测试note}

    \enote{note测试}

    \redenote{测试note}

    \inputBibliography
    \appendix
    % 附录
    \chapter{附录测试1}
    正文环境测试:滚滚长江东逝水,浪花淘尽英雄。是非成败转头空,青山依旧在,几度夕阳红。白发渔樵江渚上,惯看秋月春风。一壶浊酒喜相逢,古今多少事,都付笑谈中。
    \section{节测试}
    正文环境测试:滚滚长江东逝水,浪花淘尽英雄。是非成败转头空,青山依旧在,几度夕阳红。白发渔樵江渚上,惯看秋月春风。一壶浊酒喜相逢,古今多少事,都付笑谈中。
    \subsection{子节测试}
    正文环境测试:滚滚长江东逝水,浪花淘尽英雄。是非成败转头空,青山依旧在,几度夕阳红。白发渔樵江渚上,惯看秋月春风。一壶浊酒喜相逢,古今多少事,都付笑谈中。
    \chapter{附录测试2}
    正文环境测试:滚滚长江东逝水,浪花淘尽英雄。是非成败转头空,青山依旧在,几度夕阳红。白发渔樵江渚上,惯看秋月春风。一壶浊酒喜相逢,古今多少事,都付笑谈中。

\end{document}
