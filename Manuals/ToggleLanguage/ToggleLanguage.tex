\documentclass{../../PublicResources/DocClass}

    \DocumentTitle{「切换语言」使用手册}
    \DocumentSubtitle{Blender插件}
    \DocumentCreatedDate{2020/6/5}

    \LinkBlogPost{https://mister-kin.github.io/manuals/toggle-language/}
    \LinkPDFSource{https://github.com/Mister-Kin/OpenDocs/tree/master/Manuals/ToggleLanguage}
    \LinkVideo{}

    \AuthorName{Mr. Kin}
    \AuthorEmail{im.misterkin@gmail.com}
    \AuthorBlog{https://mister-kin.github.io/}

\begin{document}
    \pagenumbering{Roman} % 大写罗马字母样式页码。
    \maketitle
    \addcontentsline{toc}{chapter}{封面}
    \frontmatter
    \phantomsection
\begin{center}
    {\bfseries\sffamily\Large 关于作者}
\end{center}
\addcontentsline{toc}{chapter}{关于作者}

\subsection*{\bfseries \sffamily 关于我}
\begin{wrapfigure}[3]{L}{60pt}
    \vspace*{-20pt}
    \centering
    \includegraphics{kin-logo}
\end{wrapfigure}
\textbf{Mr. Kin},广东客家仁,程序猿,CG和游戏爱好者,一枚极客。翻译UP主,个人UP主。不定时在B站直播日常:码代码,码博客,翻译,做视频,做教程。 ($\vartheta$$\bullet$\_$\bullet$)$\vartheta$ \hyperlink{follow}{\emph{(点击关注我!)}}

\subsection*{\bfseries \sffamily 开源建设}

\noindent {\bfseries \sffamily 开源软件的中文化翻译}

\begin{itemize}
    \item \href{https://docs.krita.org/zh_CN/}{Krita手册}:2018.8.5 - \href{https://crowdin.com/profile}{2019.4.23}
    \item \href{https://docs.blender.org/manual/zh-hans/latest/}{Blender手册}:2019.7.20 - \href{https://www.blendercn.org/5812.html?tdsourcetag=s_pctim_aiomsg}{2019.9.4} - 至今(\href{https://developer.blender.org/p/Mr_Kin/}{翻译维护})
\end{itemize}

\subsection*{\bfseries \sffamily \hypertarget{contact}{联系方式}}
\vspace*{-1ex}
\noindent {\footnotesize \color{red} \em 注:联系时请注明身份,谢谢!}

\begin{itemize}
    \item QQ:\href{tencent://AddContact/?fromId=45&fromSubId=1&subcmd=all&uin=2312463626&website=www.oicqzone.com}{2312463626}\emph{\color{red}(点击号码加好友)}
    \item 邮箱:2312463626@qq.com ; im.misterkin@gmail.com
\end{itemize}

\subsection*{\bfseries \sffamily \hypertarget{follow}{关注渠道}}
\vspace*{-1ex}
\noindent {\footnotesize \color{red} \em 注:点击文字即可跳转关注!}
\vspace*{-2ex}

\begin{figure}[htbp]
    \centering
    \includegraphics[scale=0.2]{WechatOfficialAccounts.png}
\end{figure}
\vspace*{-4ex}

\begin{figure}[htbp]
    \centering
    \begin{minipage}[t]{0.2\textwidth}
        \centering
        \caption*{\href{https://mister-kin.github.io}{博客 - Blog}}
        \includegraphics[scale=0.055]{Blog}
    \end{minipage}
    \qquad
    \begin{minipage}[t]{0.2\textwidth}
        \centering
        \caption*{\href{https://github.com/mister-kin}{Github}}
        \includegraphics[scale=0.055]{Github}
    \end{minipage}
    \qquad
    \begin{minipage}[t]{0.2\textwidth}
        \centering
        \caption*{\href{https://weibo.com/6270111192/profile?topnav=1&wvr=6&is_all=1}{微博 - Weibo}}
        \includegraphics[scale=0.055]{Weibo}
    \end{minipage}
    \qquad
    \begin{minipage}[t]{0.2\textwidth}
        \centering
        \caption*{\href{https://www.zhihu.com/people/drwu-94}{知乎 - Zhihu}}
        \includegraphics[scale=0.055]{Zhihu}
    \end{minipage}

    \vspace*{3ex}

    \begin{minipage}[t]{0.2\textwidth}
        \centering
        \caption*{\href{http://space.bilibili.com/17025250?}{B站 - Bilibili}}
        \includegraphics[scale=0.055]{Bilibili}
    \end{minipage}
    \qquad
    \begin{minipage}[t]{0.2\textwidth}
        \centering
        \caption*{\href{http://i.youku.com/i/UNjA3MTk5Mjgw?spm=a2hzp.8253869.0.0}{优酷 - Youku}}
        \includegraphics[scale=0.055]{Youku}
    \end{minipage}
    \qquad
    \begin{minipage}[t]{0.2\textwidth}
        \centering
        \caption*{\href{https://www.toutiao.com/c/user/835254071079053/\#mid=1663279303982091}{头条 - Headline}}
        \includegraphics[scale=0.055]{Headline}
    \end{minipage}
    \qquad
    \begin{minipage}[t]{0.2\textwidth}
        \centering
        \caption*{\href{https://www.youtube.com/channel/UCXqjfWLzMlRKxGC8syWj17Q?view_as=public}{油管 - Youtube}}
        \includegraphics[scale=0.055]{Youtube}
    \end{minipage}
\end{figure}
 % 出于特殊的安全设置,\include 命令无法使用相对路径,因为需要读写权限以给 included file 写 aux 文件,而 \input 命令只需要读权限。
    \clearpage
    % !TEX root = English.tex %指定主文件。
\ifx\collections\undefined
% !TEX program = xelatex %指定编译方式xelatex。
% !BIB program = biber %指定bib数据后台处理程序biber。
\documentclass[11pt,a4paper,UTF8,titlepage]{ctexrep} %指定ctexart文档类,设置基本字号为11pt,a4大小,使用UTF-8编码保存,指定\maketitle生成单独的标题页。

\usepackage{syntonly}
%\syntaxonly %用来快速编译以排查错误,不生成DVI或PDF。
\usepackage[style=gb7714-2015]{biblatex} %调用biblatex宏包,设置参考文献样式(符合中文文献著录标准GB/T 7714-2015的样式),使用默认的后端程序biber(其支持更好,包括UTF-8等)放弃使用[backend=bibtex](只支持ascii编码)。
\addbibresource{resources/reference.bib} %加载参考文献数据库。
\usepackage{makeidx}
\makeindex %开启索引收集
\usepackage[margin=1in]{geometry} %调用geometry宏包,设置周围页边距为1英寸(为电子档设计,非打印)。
\usepackage{xcolor} %调用xcolor宏包,以支持扩展生成颜色。
\usepackage{fontspec} %调用fontspec宏包以更改西文字体族。
\setmainfont{Source Serif Pro}
\setsansfont{Source Sans Pro}
\setmonofont{Source Code Pro}
\usepackage{xeCJK} %调用xeCJK宏包以更改中文字体族。
\xeCJKsetup{AutoFakeSlant=true}  %设置xeCJK选项-伪斜体(需置于字体设置之前)。
\setCJKmainfont{思源宋体}
\setCJKsansfont{思源黑体}
\setCJKmonofont{思源等宽}
\usepackage{graphicx} %调用graphicx宏包,以支持插图。
\graphicspath{{resources/images/},{resources/images/FollowMe/}} %加载图片路径。
\usepackage{caption} %调用caption,支持不带编号的标题。
\usepackage{wrapfig} %调用wrapfig,支持图文排版。
\usepackage{subfigure} %调用subfigure宏包进行图片图片排版。
\usepackage{tikz,tikz-qtree} %调用tikz及其扩展宏包,以支持画图。
\usepackage[subfigure]{tocloft} %tocloft与subfigure宏包冲突,不能简单调用,tocloft需设置参数。
\usepackage{tocbibind} %调用宏包以添加目录本身和参考文献进目录中。
\usepackage{multicol} %调用multicol宏包以支持多栏排版。
\usepackage[toc]{multitoc} %调用multitoc宏包,设置toc目录页,默认双栏排版。
\usepackage{enumitem} %调用emuitem,以设置列表环境。
\usepackage{multirow} %调用multirow,以支持纵向合并列表。
%重订制目录命令。
\renewcommand{\tableofcontents}%
  {\chapter{\contentsname}%
  \@mkboth{\MakeUppercase\contentsname}{\MakeUppercase\contentsname}%
  \@makeschapterhead{\sourcecodename}%
  \@starttoc{toc}%
}
\usepackage{fancyhdr} %调用宏包,以设置页眉,统一格式:标题在左,页码在右。
\pagestyle{fancy}
\fancyhf{}
\fancyhead[LO]{\sffamily \rightmark}
\fancyhead[LE]{\sffamily \leftmark}
\fancyhead[ROE]{\bfseries \thepage}
\fancyfoot[COE]{\ttfamily \href{https://mister-kin.github.io}{个人博客:https://mister-kin.github.io}}
%定义intro环境。
\newenvironment{intro}{\narrower\sffamily}{\par\vspace*{2ex plus 2.5ex minus 1.5ex}}
\usepackage{listings} % 指定listings,订制代码排版环境
\lstset{
    basicstyle      = \ttfamily,                          % 基本代码指定等宽字体
    keywordstyle    = \bfseries,                          % 关键字指定加粗
    commentstyle    = \ttfamily\slshape\color{gray},      % 注释指定灰色等宽斜体
    stringstyle     = \ttfamily,                          % 字符串指定等宽字体
    %numbers        = left,                               % 行号的位置在左边,启用后不方便复制代码
    %numberstyle    = \ttfamily,                          % 行号等宽字体
    %xleftmargin    = \parindent,                         % 代码左边框起始位置(启用行号时建议启用这个)
    %frame          = trBL,                               % 代码框类型,t下,r右,b下,l左,大写时为两条线。
    %frameround     = fttt,                               % 控制代码框是否为圆角
    frameshape      = {{ryrynyyyy}{yny}{yny}{ryrynyyyy}}, % 控制边框样式,上下边是每三个字母段控制一条边框。
    backgroundcolor = \color{gray!5},                     % 代码框背景颜色:5%的灰色
    breaklines      = true,                               % 代码过长时则换行
    gobble          = 8,                                  % 去掉代码前的缩进
}
\usepackage{hyperref} %调用hyperref宏包
\hypersetup{
    colorlinks=true, %设置超链接文件带颜色
    bookmarks=true, %生成书签
    bookmarksopen=true, %书签展开
    bookmarksnumbered=true, %书签带章节编号
    CJKbookmarks=true, %cjk必设参数
    unicode, %utf-8编码必设参数
    pdftitle=标题, %设置PDF文件属性标题
    pdfauthor=Mr. Kin, %设置PDF文件属性作者
    pdfstartview=FitH %默认适合宽度显示
}

    \title{\hypertarget{title}{\textbf{标题}}}
    \addcontentsline{toc}{chapter}{标题页}
    \author{Written by Mr. Kin}
    \date{创建于2020.1.28,修改于\number\year.\number\month.\number\day}

\begin{document}
    \phantomsection %确保目录中的超链接指向正确的页码
    \pdfbookmark[1]{标题页}{title} %添加标题页书签
    \pagenumbering{Roman} %大写罗马样式页码
    \maketitle %生成标题页
    \pagenumbering{roman} %小写罗马样式页码
    {\centering \tableofcontents} %生成目录页
    \clearpage %新建页,分离上下两个样式页码的效果
    \pagenumbering{arabic} %阿拉伯样式页码
    \fi

    \phantomsection
    \section*{\bfseries \sffamily 版权声明}
    \addcontentsline{toc}{chapter}{版权声明}
    \markright{版权声明}
    \noindent 作者:Mr. Kin \\
    博文链接:\href{}{跳转博文页}\\
    %相关视频创作链接:\href{}{跳转视频页}\\ %无视频则注释掉此行
    PDF及Tex源码链接:\href{https://github.com/Mister-Kin/OpenDocs/tree/master/LearningNotes/English}{跳转PDF及Tex源码页}\\
    许可协议:本作品的所有内容,除个人设计创作的图像(如logo等)和相关的视频创作及其他特别声明外,均采用\href{https://creativecommons.org/licenses/by-nc-sa/4.0/deed.zh}{知识共享\ 署名-非商业性使用-相同方式共享 4.0 国际许可协议}进行发布。版权 © Mr. Kin,保留所有权利。\includegraphics[scale=.4]{CC-BY-NC-SA}\\*[1.3ex]
    \begin{tabular}{|*{3}{p{0.306\textwidth}|}}
        \hline
        \textsf{\bfseries 允许} & \textsf{\bfseries 限制} & \textsf{\bfseries 条件} \\
        \hline
        \vspace{-8pt}{\color{green}√} 修改 & \vspace{-8pt}{\color{red}×} 商标使用 & \vspace{-8pt}{\color{blue}$\odot$} 保留原署名 \\[-12pt]
        {\color{green}√} 分发 & {\color{red}×} 专利使用 & {\color{blue}$\odot$} 状态变更说明 \\[-12pt]
        {\color{green}√} 个人使用 & {\color{red}×} 商业使用 & {\color{blue}$\odot$} 相同的许可和版权声明 \\
        \hline
    \end{tabular}
    \\*[1.3ex]
    \emph{注:若想对本作品进行转载、引用亦或是进行二次创作时,请详细阅读上述相关协议内容(若不理解,请点击链接跳转阅读)。为保障本人权利,对于违反者,本人将依法予以处理!望周知!——Mr. Kin}
    \section*{\bfseries \sffamily 勘误声明}
    虽本人写作时已尽力保证其内容的正确性,但因个人知识面和经验的局限性以及计算机技术等相关技术日新月异,本作品内容或存在一些错误之处。还望诸君发现错误后能够\hyperlink{contact}{联系我}以更正错误,不胜感激!——Mr. Kin

    \section*{\bfseries \sffamily 侵权声明}
    若本作品采用的第三方内容侵犯了你的版权,请与我\hyperlink{contact}{联系}进行处理,谢谢!——Mr. Kin

    \section*{\bfseries \sffamily 第三方开源许可声明}
    本作品使用的第三方开源产品有:
    \begin{multicols}{2}
    \begin{itemize}
        \item \href{https://github.com/adobe-fonts}{Adobe Fonts}: \href{http://scripts.sil.org/cms/scripts/page.php?site_id=nrsi&id=OFL}{OFL v1.1}
        \item \href{https://tug.org/texlive/}{Tex Live}: \href{https://tug.org/texlive/copying.html}{TeX Live Licensing}
        \item \href{https://code.visualstudio.com/}{Visual Studio Code}: \href{https://www.mit-license.org/}{MIT}
        \item \href{http://ffmpeg.org/}{FFmpeg}: \href{http://ffmpeg.org/legal.html}{LGPL v2.1}
        \item \href{https://krita.org/en/}{Krita}: \href{https://docs.krita.org/en/KritaFAQ.html?highlight=license#license-rights-and-the-krita-foundation}{Krita's GPL license}
        \item \href{https://inkscape.org/}{Inkscape}: \href{https://inkscape.org/about/license/}{GPL}
        \item \href{https://www.gimp.org}{GIMP}: \href{https://www.gimp.org/about/COPYING}{GPL}
        \item \href{https://www.blender.org}{Blender}: \href{https://www.blender.org/about/license/}{GPL}
        \item \href{https://www.audacityteam.org/}{Audacity}: \href{https://www.gnu.org/licenses/old-licenses/gpl-2.0.en.html}{GPL v2}
        \item \href{https://handbrake.fr}{Handbrake}: \href{https://github.com/HandBrake/HandBrake/blob/master/LICENSE}{GPL v2}
    \end{itemize}
    \end{multicols}

    \ifx\collections\undefined
    \printbibliography %生成参考文献排版。
    \addcontentsline{toc}{chapter}{参考文献} %添加参考文献进目录
    \clearpage %新建页,确保超链接跳转正确
    \phantomsection %确保目录中的超链接指向正确的页码
    \printindex %生成索引排版。
\end{document}
    \fi

    \clearpage
    {\centering \tableofcontents} % 生成目录页。
    \mainmatter

    % 正文
    \chapter{切换语言}
    \section{介绍}
    一款blender插件,旨在通过一键快速、轻松地在两种语言之间切换用户界面,而非重复打开偏好设置。

    \subsection{背景}
    在早期进行blender手册翻译工作时,因需要校对相关UI内容,本人频繁地在中英文之间切换软件的用户界面语言。但每次繁琐地使用偏好设置进行语言切换令人苦不堪言,索性就开发一款插件以简化这操作。

    \subsection{面向人群}
    \begin{itemize}
        \item 翻译者:需要频繁切换界面语言以校对相关内容。
        \item CG学习者:需要频繁切换界面语言才能跟进外语教程。
    \end{itemize}

    \subsection{插件目前实现的功能}
    \begin{itemize}
        \item 一键切换UI语言(目前仅支持:简中/英)
        \item 一键打开用户偏好设置
        \item 插件的个性化设置菜单
        \item 一键设置自定义的「用户偏好设置」({\color{red}该功能还处于试验性中,慎用})
    \end{itemize}
    \noindent {\footnotesize \emph{注:更多详细介绍请查看「\hyperlink{AddonFeatures}{插件功能}」小节内容,规划中的功能请跳转\href{https://mister-kin.github.io/roadmap/\#切换语言-Blender-插件}{蓝图规划页}查看。}}

    \section{用法}

    \subsection{下载及安装}
    \noindent {\bfseries \sffamily 下载步骤}\par
    \begin{itemize}
        \item 下载地址:\href{https://github.com/Mister-Kin/ToggleLanguage/releases/latest}{点击跳转},下载ToggleLanguage.zip文件。
        \item blender版本要求:\href{https://www.blender.org/download/}{v2.83+}
    \end{itemize}

    \noindent {\bfseries \sffamily 安装步骤}
    \begin{enumerate}
        \item 运行Blender。
        \item 打开「Preferences/用户偏好设置」(Menu/菜单>Edit/编辑>Preferences/用户偏好设置)。
        \item 选择「Add-ons/插件」选项卡。
        \item 点击「Install.../安装」后,选择先前所下载好的ToggleLanguage.zip并点击确定。
        \item 启用插件。
    \end{enumerate}

    \begin{figure}[htbp]
        \begin{minipage}[t]{0.45\textwidth}
            \includegraphics[scale=0.26]{Installation}
            \caption{安装方法}
        \end{minipage}
        \qquad
        \begin{minipage}[t]{0.45\textwidth}
            \includegraphics[scale=0.26]{EnableAddon}
            \caption{启用插件}
        \end{minipage}
    \end{figure}

    \subsection{插件UI}
    \label{插件UI小节}
    \begin{figure}[htbp]
        \includegraphics[scale=0.24]{UI}
        \caption{插件UI(图示菜单已被下拉展开)}\label{插件UI}
    \end{figure}

    如图\ref{插件UI}所示,启用插件后,可看见插件位于顶端菜单栏末尾处。插件UI从左往右看,有三个元素:
    \begin{enumerate}
        \item 左侧是语言切换按钮。
        \item 中间是插件的个性化设置菜单。
        \item 右侧是用户偏好设置按钮。
    \end{enumerate}

    \subsection{插件功能}
    \hypertarget{AddonFeatures}{}
    \noindent {\footnotesize \em 注:以下按照UI排布位置进行功能介绍。UI排布位置详见\ref{插件UI小节}小节。}
    \begin{enumerate}
        \item 语言切换按钮(快捷键F5):切换blender用户界面语言(目前仅支持:简中/英)。
        \item 设置菜单:插件的个性化设置。
        \begin{enumerate}
            \item UI提示方案菜单:选择相应UI提示的方案。
            \begin{enumerate}
                \item 默认模式:关闭「开发选项」「Python工具提示」,如图\ref{UI提示方案菜单:默认模式}所示。
                \item 开发者模式:启用「开发选项」「Python工具提示」,如图\ref{UI提示方案菜单:开发者模式}所示。
            \end{enumerate}
            \item 新建数据-翻译按钮:启用/关闭新建数据的翻译功能,默认值为关闭,如图\ref{新建数据选项}所示。在非英语界面环境中,启用该功能可使blender新建数据的名称为当前语言,如图\ref{新建数据效果}所示,新建平面的名称为「平面」,而非「plane」。
            \item 我的偏好设置按钮({\color{red}试验性,慎用}):部署我个人的偏好设置。所涉及的设置选项详见\hyperlink{MyPreferences}{我的偏好设置}。(注意:使用该功能,同时会将一些设置保存进启动文件中。然而重置偏好设置按钮无法重置启动文件,\marginpar{\scriptsize 在测试中,使用插件来重置启动文件会导致blender闪退,故只保留重置偏好设置的功能。}如果需要还原所有设置,除了使用插件的重置按钮,仍需使用「Menu/菜单>File/文件>Default/默认>Load Factory Settings/加载初始设置」以重置启动文件。)
            \item 重置偏好设置按钮:重置偏好设置。
        \end{enumerate}
        \item 用户偏好设置按钮(快捷键Ctrl+Alt+U):打开用户偏好设置窗口。
        \item P.S.搜索功能增设快捷键F6(因本人有其他软件占用全局快捷键F3,故增设F6键)
    \end{enumerate}

    \begin{figure}[htbp]
        \begin{minipage}[t]{0.45\textwidth}
            \includegraphics[scale=0.26]{DefaultHint}
            \caption{UI提示方案菜单:默认模式}
            \label{UI提示方案菜单:默认模式}
        \end{minipage}
        \qquad
        \begin{minipage}[t]{0.45\textwidth}
            \includegraphics[scale=0.26]{DeveloperHint}
            \caption{UI提示方案菜单:开发者模式}
            \label{UI提示方案菜单:开发者模式}
        \end{minipage}

        \vspace{1ex}

        \begin{minipage}[t]{0.38\textwidth}
            \includegraphics[scale=0.20]{NewDataOption}
            \caption{新建数据-翻译按钮启用/关闭的选项}
            \label{新建数据选项}
        \end{minipage}
        \begin{minipage}[t]{0.65\textwidth}
            \includegraphics[scale=0.18]{NewDataEffect}
            \caption{新建数据-翻译按钮启用后的效果:图示界面语言为简中}
            \label{新建数据效果}
        \end{minipage}
    \end{figure}

    \newpage
    \noindent {\bfseries \sffamily \hypertarget{MyPreferences}{我的偏好设置}(我本人的一些设置习惯)}\par
    \begin{enumerate}
        \item 偏好设置部分:
        \begin{enumerate}
            \item 界面>分辨率缩放到1.3
            \item 主题> Blender Light
            \item 插件>启用插件Node Wrangler、Cell Fracture、Auto Tile Size、Development: Icon Viewer
            \item 视图切换>启用「围绕选择物体选择」和「缩放至鼠标位置」
            \item 键位映射>启用「Pie Menu on Drag」和「Extra Shading Pie Menu Items」
            \item 系统
            \begin{enumerate}
                \item Cycles渲染设备> CUDA
                \item 声音>音频设备> SDL
            \end{enumerate}
            \item 报错\&加载>启用「压缩文件」、关闭「自动保存」
            \item 文件路径
            \begin{enumerate}
                \item 纹理:H:\textbackslash Textures\textbackslash
                \item 临时文件:E:\textbackslash Temp\textbackslash
                \item 渲染输出:E:\textbackslash Process\textbackslash
                \item 渲染缓存:E:\textbackslash Temp\textbackslash
            \end{enumerate}
        \end{enumerate}
        \item 启动界面部分(注意:使用「我的偏好设置」功能,会同时将以下设置保存进启动界面文件):
        \begin{enumerate}
            \item 属性编辑器
            \begin{enumerate}
                \item 渲染属性
                \begin{enumerate}
                    \item 渲染引擎> Cycles
                    \item 设备> GPU计算
                    \item 采样>启用「自适应采样」
                    \item 性能> Auto Tile Size > Target Tile Size > 256
                    \item 性能>线程>多线程模式>固定
                    \item 性能>线程>线程> 6
                \end{enumerate}
                \item 输出属性>输出路径:E:\textbackslash Process\textbackslash
            \end{enumerate}
        \end{enumerate}
        \item 状态栏>启用「场景统计数据」「系统内存」「显存」
    \end{enumerate}

    \section{开发记录}
    \subsection{开发测试环境}
    \begin{itemize}
        \item Win10 v20H2 OS
        \item Blender v2.92
    \end{itemize}

    \noindent {\footnotesize \emph{注:建议在 blender v2.83 以上使用。2.83默认是启用翻译功能,2.83以下版本的翻译都得手动启用。因为没有实际测试过,所以2.83以下版本要使用该插件,可能需要手动启用翻译功能。这也是我将插件最低支持版本改为2.83的原因,当然,你也可在2.83以下的版本使用。}

    \newpage
    \subsection{更新历史}
    \begin{table}[!h]
        \centering
    \begin{tabular}{|*{2}{c|}p{240pt}|c|}
        \hline
        版本 & 更新日期 & \centering{更新内容} & 标签 \\
        \hline
        v0.6 & 2021/3/10 & 添加布尔值按钮-新数据翻译;支持快捷键;代码重构,插件翻译重构…… & Features \\
        \hline
        v0.5 & 2020/12/8 & 支持Blender v2.91;支持从.zip文件安装插件…… & Changes \\
        \hline
        v0.5-beta & 2020/9/20 & 添加菜单用以选择UI提示方案 & Features\\
        \hline
        v0.4 & 2020/8/3 & 代码项目重命名为ToggleLanguage;按钮UI放置于最右端;添加新按钮以快速查看preferences & Features \\
        \hline
        v0.3 & 2020/6/5 & 更新至blender 2.83 api;修改部分类名;添加文档链接;修改完善许可说明 & Minor changes \\
        \hline
        v0.2 & 2020/5/21 & 清除未使用属性的报错 & Bug fixes \\
        \hline
        v0.1 & 2020/5/12 & 完成基础的功能设计「一键切换」 & Features \\
        \hline
    \end{tabular}
    \end{table}

    \noindent {\footnotesize \em 注:更多详细的信息,请查看\href{https://github.com/Mister-Kin/ToggleLanguage/releases}{Release Notes}。}

    \section{作者}
    \textbf{ToggleLanguage} © Mr. Kin, 项目代码采用 \href{https://github.com/Mister-Kin/ToggleLanguage/blob/master/LICENSE}{GNU GPL v3.0} 许可协议进行发布。

    由 Mr. Kin 著作并维护。

    \nocite{*} % 不使用 cite 也能生成参考文献。
    \printbibliography % 生成参考文献排版。
    \addcontentsline{toc}{chapter}{参考文献} % 添加参考文献进目录。

    \appendix
    % 附录

\end{document}
